\documentclass[11pt,a4paper]{moderncv}
\moderncvtheme[green]{classic}
\usepackage[utf8]{inputenc}
\usepackage[top=1.5cm, bottom=1.0cm, left=2cm, right=2cm]{geometry}
\setlength{\hintscolumnwidth}{3cm}

\firstname{Nathan}
\familyname{AGUESSE}
\title{Étudiant - Développeur Informatique}
\address{2 résidence les Grands Arbres}{95130 Le Plessis-Bouchard}
\mobile{07 61 43 95 36}
\email{aguessenathan@gmail.com}
\extrainfo{18 ans}


\begin{document}
\maketitle


\section{Formations}

\cventry{Septembre 2021\\à Aujourd'hui}{Licence Informatique}{Université Paris 8}{}{}{Informatique générale avec Parcours mineur "Conception et Programmation de Jeux-Vidéo"}


\section{Diplomes}

\cventry{Juillet 2018}{Diplôme Nationalle du Brevet}{Collège les Châtenades}{}{}{Mention Bien}

\cventry{Juillet 2021}{Baccalauréat}{Lycée Arnaut Daniel}{}{}{Mention Assez bien}


\section{Avenir}


\cventry{Prochainement...}{Candidater pour le Master JMIN}{École CNAM-ENJMIN}{}{}{Master Jeux et Médiax Intéractifs Numériques, Parcours "Programmation"}


\section{Projets}

\cventry{Juillet 2019}{Shmup}{Python}{}{}{Projet Final d'ICN}

\cventry{Juillet 2020}{Bricks Breakers}{Processing}{}{}{Projet Final d'NSI}

\cventry{Décembre 2021}{Bubble Tank}{Godot - Visual Script}{}{}{Projet d'Introduction au Moteurs de Jeux}

\cventry{Décembre 2021}{Siteweb}{HTML/CSS}{}{}{Projet de Gestion d'Identité en Ligne}

\cventry{Janvier 2022}{Othello}{Python}{}{}{Projet de Méthodologie de la Programmation}


\section{Compétences}

\cvitem{\underline{Programmations}}{Python, HTML/CSS, Processing}
\cvitem{\underline{Applications}}{VSCode, Godot}
\cvitem{\underline{Langues}}{Français Natale, Anglais B1, Japonais Débutant}


\section{Qualités}

\cvitem{}{Persévérant, Curieux, Attentif, Discret, Flexible}


\section{Passions}

\cvitem{}{Anime/Manga, Japon, Programmation, Jeux-Vidéo}

\end{document}